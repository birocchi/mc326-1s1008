\documentclass[a4paper,10pt]{article}
\usepackage[brazil]{babel}
\usepackage[utf8]{inputenc}

\title{\textbf{Relatório TP4}}
\author{\textbf{Grupo 1} \\
  Anderson Phillip Birocchi (072787) \\
  Miguel Francisco Alves de Mattos Gaiowski (076116) \\
  Raphael Kubo da Costa (072201)}
\date{\today}
\begin{document}

\maketitle

\begin{abstract}
Este trabalho prático possui dois objetivos principais: implementar uma camada de \textit{hashing} para os índices mantidos em memória até o Trabalho Prático anterior, de modo a dividi-los em vários arquivos e carregar somente um deles em memória de cada vez, e implementar uma busca de imagens por similaridade de conteúdo.

Este relatório detalha algumas decisões de design e arquitetura feitas durante o desenvolvimento, além de uma descrição do código fonte e um guia de uso do programa.
\end{abstract}

\section{Alterações}
Dentre as principais alterações no código e nas funcionalidades do programa em relação ao trabalho prático anterior, podemos destacar:\\

\textbf{Funcionalidades}:
\begin{itemize}
\item Adicionada a opção de busca por similaridade de conteúdo entre imagens;
\item O NRR de cada entrada exibida na listagem em HTML não é mais exibido;
\item A opção de listagem de todas as obras foi removida.
\end{itemize}

\textbf{Código}:
\begin{itemize}
\item A estrutura \texttt{MemoryIndex} foi adaptada para ter seu conteúdo em disco dividido em diversos arquivos ao invés de apenas um. Para indicar em qual arquivo determinada entrada está, é utilizada uma função de \textit{hashing} que utiliza o nome da entrada como chave;
\item A função \texttt{memory\_index\_load\_from\_file} foi tornada \textit{static} e renomeada para \texttt{load\_file};
\item Implementadas funções auxiliares em \texttt{file.c} para a criação e checagem de validade de arquivos, como \texttt{file\_create\_if\_needed} e \texttt{file\_is\_valid\_image};
\item As funções \texttt{secindex\_insert\_wrapper} e \texttt{secindex\_remove\_wrapper} foram unificadas em \texttt{secindex\_wrapper}, com o \textit{enum} \texttt{SecIndexWrapperAction} definindo a ação;
\item Criadas uma estrutura \texttt{Descriptor} e as estruturas auxiliares \texttt{SimiliarityList} e \texttt{SimilarityRecord}, para o cadastro das informações do descritor de cada imagem do banco de dados;
\item Boa parte do código da função \texttt{adapter\_find} foi reescrito para suportar de forma limpa e eficiente as três estruturas usadas em busca: \texttt{MemoryIndex}, \texttt{SecondaryIndex} e \texttt{Descriptor}.
\end{itemize}

\section{Uso da interface}
O programa gerado pelo \textit{Makefile} chama-se \textbf{tp4} e deve ser rodado a partir de um terminal. Seu menu inicial apresenta as seguintes opções: \textbf{inserir} uma nova entrada na base de dados, \textbf{consultar} uma obra existente no banco de dados, \textbf{remover} uma obra e \textbf{sair}.

A opção é selecionada digitando-se \textbf{i} para inserir, \textbf{c} para consultar, \textbf{r} para remover e \textbf{s} para sair.

\subsection{Inserção}
No modo de \textbf{inserção}, o usuário deve digitar as informações para \textit{nome da obra}, \textit{tipo de obra}, \textit{autor da obra}, \textit{ano em que a obra foi feita}, \textit{valor da obra} e um \textit{identificador da obra}. Determinados campos possuem restrições: ano e valor devem conter apenas dígitos, e o identificador precisa terminar nas extensões \textit{png, jpg} ou \textit{gif} e ser precedido por até quatro dígitos. Caso já exista uma obra com mesmo nome na base de dados, o usuário deve entrar com outros dados de obra. Feita a inserção, o usuário pode escolher inserir uma nova obra ou voltar para o menu principal.

\subsection{Consulta}
No modo de \textbf{consulta}, o usuário é apresentado às formas de consulta que podem ser realizadas: \textit{parcial}, \textit{exata} ou por \textit{similaridade}.

Uma busca \textit{parcial} é feita somente nos índices secundários do banco de dados. É possível buscar por \textbf{ano}, \textbf{tipo}, \textbf{título} ou \textbf{autor} da obra. Neste modo de pesquisa, é possível buscar por apenas \textbf{uma} palavra, que retornará as entradas que contiverem a palavra buscada. O programa não faz distinção entre maiúsculas e minúsculas.

Uma busca \textit{exata} é feita somente no índice primário (título da obra), e busca o título exato entrado pelo usuário. Neste caso, também não há distinção entre maiúsculas e minúsculas.

A busca por \textit{similaridade} possui outra interface. Inicialmente, o usuário deve entrar com o caminho completo de uma imagem que será usada como base de comparação. Caso a imagem não seja encontrada, ou se não possuir uma das extensões aceitas pelo programa (png, jpg ou gif), uma mensagem de erro é exibida e o usuário é levado novamente ao menu inicial. Se a imagem for um arquivo válido, o programa pergunta ao usuário o número máximo de imagens semelhantes devem ser exibidas em listagem. O programa realiza, então, uma busca em seu banco de dados de imagens por imagens semelhantes à base de comparação e, em seguida, lista as ocorrências juntamente com a imagem base.

Os resultados encontrados são escritos com os campos título, tipo, autor, ano de criação, valor e identificador no arquivo \texttt{lista.html}. Caso não haja nenhum resultado, exibe-se uma mensagem de erro.

\subsection{Remoção}
No modo de \textbf{remoção}, o usuário deve entrar com o título completo da obra que deseja excluir do banco de dados. O programa não realiza distinções entre maísculas e minúculas. O título pode ser obtido através de uma \textbf{consulta} de obras. Quando o título digitado pelo usuário é inválido, é exibida uma mensagem de erro e o programa retorna ao menu inicial.

\section{Descrição do Trabalho}
Em relação à versão anterior do trabalho, não foram feitas grandes alterações na estrutura ou na quantidade de arquivos presentes, ao contrário do que ocorreu na transição entre os trabalhos práticos 2 e 3.

Foram criados apenas os arquivos \texttt{descriptor.c}, \texttt{descriptor.h}, \texttt{hash.c} e \texttt{hash.h}, enquanto os outros arquivos presentes tiveram apenas algumas funcionalidades editadas. Nenhum arquivo de código da versão anterior foi removido. Atualmente, os seguintes arquivos compõem o programa:

\begin{description}
\item[adapter.c, adapter.h:] \textit{Design pattern} utilizada como uma espécie de ``cola'' entre as diversas partes do sistema: coordena índices e base para buscas, remoções e inserções.

\item[avail.c, avail.h:] Contêm as funções referentes à manipulação das avail lists usadas pelos índices e pela base de dados.

\item[base.c, base.h:] Contêm as funções referentes à manipulação da base de dados e definições relacionadas, como o tamanho dos campos da base de dados e a estrutura que representa uma entrada na base de dados. Há funções para adição e remoção na base de dados, além de funções auxiliares que validam o identificador e lêem entradas na base de dados.

\item[descriptor.c, descriptor.h:] Contêm as funções referentes à manipulação do índice de descritores de imagens, incluindo adição, remoção e busca no índice.

 \item[file.c, file.h:] Contêm funções relacionadas à manipulação de arquivos quaisquer: checa se um arquivo existe, se possui uma extensão válida, se não está vazio.

\item[filelist.h:] Contêm os nomes dos arquivos utilizados pelo banco de dados (base, chaves primárias, secundárias e avail lists), de modo a evitar nomes \textit{hardcoded} no código. Possui também a quantidade de índices secundários existentes e o número de arquivos em que o arquivo de índice em memória foi dividido para os índices (com exceção do índice de descritores, que possui contagem própria).

\item[hash.c, hash.h:] Contêm uma função de \textit{hashing} genérica (algoritmo de Fowler/Noll/Vo) e uma função de adição do sufixo correto a um arquivo para a obtenção do arquivo de hashing correto.

\item[html.c, html.h:] Contêm as funções relacionadas à manipulação e geração de arquivos HTML para consultas e listas usadas pelo programa, para automatizar o processo de escrita de dados. Atualmente, cria o começo e o fim de um arquivo HTML e escreve as informações de uma entrada no banco de dados em formato HTML.

 \item[io.c, io.h:] Controlam a leitura e escrita de dados entrados pelo usuário. Já no Trabalho Prático 2 foi implementada uma função para limpar caracteres em branco em excesso nas strings. Neste trabalho, foram implementadas uma função para ler somente a primeira palavra de uma string e um \textit{foreach} para que uma string tenha uma função chamada para cada palavra.

 \item[main.c:] Contém a lógica de execução do programa. Inicia a base de dados e a lista de chaves primárias e depois recebe as opções de operações desejadas pelo usuário, tratando-as e chamando as funções necessárias em outros arquivos.

 \item[mem.c, mem.h:] Contêm macros e funções para facilitar a alocação de memória para o programa: basicamente, as macros alocam o número desejado de bytes na memória e já abortam o programa em caso de erro.
	
 \item[menu.c, menu.h:] Contêm as funções de exibição dos menus usados pelo programa, além de funções para análise das opções de menu entradas pelo usuário.

 \item[secindex.c, secindex.h:] Responsáveis pela manipulação dos índices secundários do banco de dados: adicionam, buscam e removem índices secundários.
\end{description}

\section{Arquivos gerados e existentes}
\begin{description}
\item[doc/api/*:] Documentação da API do sistema em HTML e LaTeX.
\item[author, type, title, year.av:] Avail list dos índices secundários de autores, títulos, tipos e anos.
\item[author, type, title, year.sk.h*:] Índices de memória escritos em disco para os índices de autores, títulos, tipos e anos.
\item[author, type, title, year.sl:] Listas \textit{entry-based} de referências à chave primária para os índices de autores, títulos, tipos e anos.
\item[base01.dat:] Base de dados utilizada pelo programa.
\item[descriptor.desc.h*]: Índice de descritores das imagens da base de dados.
\item[doxygen.conf:] Arquivo de configuração para o Doxygen.
\item[entrada.in:] Entradas usadas para popular a base dados.
\item[lista.html:] Resultado das consultas à base de dados.
\item[pkfile.pk:] Arquivo de ``cache'' das chaves primárias (título das obras).
\end{description}

\section{Decisões de design}\label{designdecisions}
Em face dos novos requisitos apresentados no enunciado deste trabalho prático, à maneira do trabalho prático anterior as primeiras atitudes tomadas foram tentar reutilizar código, encaixar as novas estruturas e os novos requisitos no código atual e, só então, criar novas estruturas.

Inicialmente, foram criados os arquivos \texttt{hash.c} e \texttt{hash.h}. Pensou-se que seriam arquivos grandes, porém devido à implementação da estrutura \texttt{MemoryIndex}, não foram precisas muitas mudanças para atender o requisito de ``quebrar'' o conteúdo da estrutura em vários arquivos ao escrever seu conteúdo em disco. Assim, os arquivos \texttt{hash.c} e \texttt{hash.h} possuem apenas duas funções: uma implementação do algoritmo FNV-1 de \textit{hashing} (ver \ref{fnv1}) e uma função de manipulação de \textit{strings} responsável por acrescentar o sufixo certo a determinado arquivo (o sufixo ``.hNNNN''). Em \texttt{memindex.c}, foi implementada a função \textit{static} \texttt{change\_hash\_file}, responsável por, dada uma chave, carregar o arquivo onde ela está contida caso ela esteja cadastrada. Foram feitas modificações pontuais nas funções de busca, adição e remoção de entradas para que todas chamassem \texttt{change\_hash\_function} quando precisassem mudar o arquivo carregado. A quantidade de arquivos de hash em que o índice deve ser dividido é a constante \texttt{INDEX\_HASH\_NUM}, definida em \texttt{filelist.h}. Editanto-a e executando-se \texttt{make clean}, é possível alterar o número de arquivos em que cada índice secundário ou primário (com exceção do descritor) será dividido.

Percebeu-se que o índice de descritores possuía grande semelhança com a estrutura atual de índices secundários, \texttt{SecondaryIndex}. Assim, a idéia inicial foi tentar criar apenas mais um índice secundário como os já existentes (ano, autor, tipo e título) que conteria o descritor e o nome da obra correspondente (a chave primária).

Entretanto, embora o índice de descritores apresente similaridades com relação aos índices secundários, sua estrutura apresentava diferenças grandes o suficiente para impedir sua implementação como um novo índice secundário sem grandes modificações no código que poderiam causar instabilidades e gerar a necessidade de testes intensivos. Como exemplo de diferença, podemos destacar a composição dupla dos índice secundários. Índices secundários são compostos por uma estrutura em memória que possui um campo único (uma parte do título de obra, ou um ano) e um número relativo de registro que aponta para a parte da estrutura residente no disco, com o nome da obra (chave primária) e um apontador para o elemento anterior da lista, como explicado em aula. Nesta implementação do trabalho prático, essas tarefas são gerenciadas pelas estruturas \texttt{SecondaryIndex} e \texttt{MemoryIndex}. A estrutura de descritores, por sua vez, possui apenas a parte em disco: as entradas no formado \textit{descritor-chave primária} são escritas de maneira não ordenada em disco nos arquivos \texttt{descriptor.desc.hNNNN}, com \texttt{NNNN} variando de \texttt{0000} a \texttt{0008}. Ora, tentar utilizar a estrutura \texttt{SecondaryIndex} para ter acesso a apenas metade de suas funcionalidades seria um erro de design e arquitetura. Por outro lado, separar a parte usada pelas duas estruturas geraria um esforço desnecessário, devido principalmente à falta de recursos de orientação a objetos na linguagem C.

Assim, foram criadas as estruturas \texttt{Descriptor}, \texttt{SimilarityList} e \texttt{SimilarityRecord}, que têm sua estrutura semelhante à de \texttt{MemoryIndex} e \texttt{MemoryRecord}, porém com comportamento diferente principalmente na inserção e busca de dados. As novas estruturas foram implementadas nos arquivos \texttt{descriptor.c} e \texttt{descriptor.h}, que também possuem uma função \textit{static} \texttt{change\_hash\_file} com implementação adaptada às necessidades das novas estruturas.

\subsection{Função de hashing FNV-1}\label{fnv1}
Para o \textit{hashing} dos índices secundários e do índice primário, foi utilizado o algoritmo FNV-1, criado por Glenn Fowler, Landon Curt Noll e Phong Vo. Trata-se de um algoritmo de \textit{hashing} bastante simples, porém muito eficiente.

Começa-se atribuindo a uma variável \textit{hash} o valor de \textit{offset\_basis}. Então, para cada byte da entrada, multiplica-se o valor de \textit{hash} por um \textit{primo FNV}, e depois realiza-se uma operação XOR com o mesmo \textit{hash}. Na implementação do algoritmo neste trabalho, o resultado final de \textit{hash} é retornado após um \textit{mod} com a constante \texttt{INDEX\_HASH\_NUM}, que contém a quantidade de arquivos em que cada índice foi dividida.

\textit{offset\_basis} e o \textit{primo FNV} são constantes dependentes do número de bits do hash final. Nesta implementação, \textit{hash} é um \texttt{unsigned int} com 32 bits de tamanho (não foram realizados testes com o programa rodando num sistema de 64 bits, mas a implementação deve continuar funcionando). Para um hash de 32 bits, o algoritmo indica o valor 2166136261 como \textit{offset\_basis} e o primo 16777619 como \textit{primo FNV}. O valor de \textit{offset\_basis} é o resultado do hash da string ``chongo $<$Landon Curt Noll$>$ /\textbackslash../\textbackslash'' com o algoritmo FNV-0, uma versão anterior do algoritmo que assumia \textit{offset\_basis} = 0. A escolha dos valores do \textit{primo FNV} não é descrita em detalhes por Noll, que diz, em sua página:

\begin{quote}
The theory behind which primes make good FNV\_primes is beyond the scope of this webpage.
\end{quote}

\section{Resultado final e comentários}
Rodando o Valgrind no binário \textit{tp4} sem o uso do \textit{Electric Fence}, não foram constatados erros ou vazamentos de memória. A consulta por similaridade e a divisão dos índices em vários arquivos funcionam como esperado.

Para controle de versões do código, desde o primeiro trabalho é utilizado o \textbf{Subversion}. O repositório do grupo encontra-se em \textit{http://code.google.com/p/mc326-1s1008}. A partir do segundo trabalho prático, passou-se a utilizar o \textit{Doxygen} para geração de documentação do código-fonte. Por fim, este relatório e seu fonte em LaTeX encontram-se na pasta dos fontes do programa.

O tempo de implementação deste trabalho foi consideravelmente menor que o gasto para o trabalho prático anterior, em parte devido ao fato das principais estruturas do programa (os índices de memória) e o \textit{Adapter} já estarem consolidados e não terem sofrido grandes alterações. Ainda assim, como evidenciado na seção \ref{designdecisions}, gastou-se muito esforço tentando unificar o índice de descritores e \texttt{MemoryIndex}, sem grandes sucessos. Futuramente, espera-se reusar mais código nessas duas estruturas.

Como não houve grandes alterações estruturais no programa, não foram encontrados tantos bugs como anteriormente. Em relação ao sistema de controle de versões, foram criados \textit{branches} para os trabalhos práticos 2 e 3, e quando havia \textit{bug fixes} importantes para alguma das versões, eles eram inseridos nos branches, possibilitando a reentrega desses trabalhos com as alterações sugeridas em correções anteriores.

\end{document}
