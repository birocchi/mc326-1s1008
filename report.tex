\documentclass{article}
\usepackage[brazil]{babel}
\usepackage[utf8]{inputenc}

\title{Relatório TP1}
\author{Anderson Phillip Birocchi \and
  Miguel Francisco Alves de Mattos Gaiowski \and
  Raphael Kubo da Costa}
\date{\today}
\begin{document}

\maketitle

Nosso projeto consiste de vários arquivos .c e .h.
O arquivo data.h é onde estão definidos os tipos usados no programa. Basicamente, é onde está definida estrutura de armazenamento dos dados. No futuro, quando o projeto crescer, mais tipos serão definidos ali.
Os arquivos io.c e io.h(cabeçalhos das funções de io.c) controlam a leitura e escrita de dados. A função readData comunica-se com o usuário e chama a função readValues que lê os dados de forma apropriada. A funçao writeData toma conta da escrita dos dados no arquivo.
Optamos por usar a função fgets para a leitura de dados da entrada padrão, já que podemos especificar o tamanho da entrada. No caso de uso de scanf o usuário poderia, por exemplo, colocar um título com mais de 200 caracteres. Devido ao uso da função fgets, tivemos que criar uma função simplória para extração do $\backslash$n no final da string retornada.
Os arquivos menu.c e menu.h carregam as funções de impressão de boas vindas e do menu de opções para o usuário. Como a impressão do menu é algo que ocorre com frequência, é saudável definir funções para tal.
O código contido em main.c 

\end{document}
