\documentclass{article}
\usepackage[brazil]{babel}
\usepackage[utf8]{inputenc}

\title{\textbf{Relatório TP2}}
\author{\textbf{Grupo 1} \\
  Anderson Phillip Birocchi (072787) \\
  Miguel Francisco Alves de Mattos Gaiowski (076116) \\
  Raphael Kubo da Costa (072201)}
\date{\today}
\begin{document}

\maketitle

\section{Objetivo}\

Neste trabalho prático deve ser implementado um sistema de consulta por chaves primarias numa base de dados. A estrutura desta base é uma modificação da base feita no Trabalho Prático 1. Além da consulta, pede-se que o usuário tenha a possibilidade de gerar uma lista de todas as obras de arte na base de dados.

Este programa deve ser capaz de fazer tudo o que o da primeira versão fazia, adicionando-se as funcionalidades supra-citadas.

\section{Uso da interface}\

O programa gerado pelo \textit{Makefile} chama-se \textbf{tp2} e deve ser rodado a partir de um terminal. Seu menu inicial apresenta quatro opções: \textbf{inserir} uma nova entrada na base de dados, \textbf{consultar} uma obra existente no banco de dados, \textbf{gerar} uma lista com informações sobre todas as obras da base de dados e \textbf{sair}.

Cada opção é selecionada digitando-se \textbf{i} para inserir, \textbf{c} para consultar, \textbf{g} para gerar a lista de obras e \textbf{s} para sair.\\

No modo de \textbf{inserção}, o usuário deve digitar as informações para \textit{nome da obra}, \textit{tipo de obra}, \textit{autor da obra}, \textit{ano em que a obra foi feita}, \textit{valor da obra} e um \textit{identificador da obra}. Determinados campos possuem restrições: ano e valor devem conter apenas dígitos, e o identificador precisa terminar nas extensões \textit{png, jpg} ou \textit{gif} e ser precedido por até quatro dígitos. Caso já exista uma obra com mesmo nome na base de dados, o usuário deve entrar com outros dados de obra. Feita a inserção, o usuário pode escolher inserir uma nova obra ou voltar para o menu principal.\\

No modo de \textbf{consulta}, o usuário deve entrar com o nome da obra cujas informações deseja obter. Deve-se tomar atenção, pois a consulta é \textit{case-sensitive}, ou seja, diferencia maiúsculas e minúsculas. Caso uma obra com o título desejado seja encontrada, suas informações (título, tipo, autor, ano de criação, valor e identificador) são escritas no arquivo \textbf{lista.html} para exibição. Se não houver nenhuma obra com esse nome no banco de dados, é exibida uma mensagem de erro.\\

No modo de \textbf{geração de lista}, escreve-se no arquivo \textbf{lista.html} uma lista com as informações (título, tipo, autor, ano de criação, valor e identificador) de todas as obras cadastradas na base de dados. Caso não haja nenhuma obra na base de dados, exibe-se uma mensagem de erro.

\section{Descrição do Trabalho}\

Desde a versão anterior foi definida uma divisão dos códigos fonte para melhor manutenção e escalabilidade. Com o crescimento das funcionalidades e necessidade de melhor base para algumas destas, acabou-se modificando a estrutura dos arquivos para melhor adequação. Atualmente, os seguintes arquivos compõem o programa:

\begin{itemize}
 \item \textbf{base.c, base.h}\\
	Contêm as funções referentes à manipulação da base de dados e definições relacionadas, como o tamanho dos campos da base de dados e a estrutura que representa uma entrada na base de dados. As funções validam o identificador, lêem uma entrada, escrevem uma entrada e manipulam o identificador para retornar um caminho válido.

 \item \textbf{file.c, file.h}\\
	Contêm funções relacionadas à manipulação de arquivos quaisquer: atualmente, apenas checa se determinado arquivo existe e retorna o tamanho de determinado arquivo.

 \item \textbf{html.c, html.h}\\
	Contêm as funções relacionadas à manipulação e geração de arquivos HTML para consultas e listas usadas pelo programa, para automatizar o processo de escrita de dados. Atualmente, cria o começo e o fim de um arquivo HTML e escreve as informações de uma entrada no banco de dados em formato HTML.

 \item \textbf{io.c, io.h}\\
	Controlam a leitura e escrita de dados entrados pelo usuário. Foram implementadas também funções para limpar os espaços em branco no começo e no fim da strings, além daqueles espaços duplicados no meio, além de uma função para ler caracteres entrados pelo usuário. Continuam todas as funções do Trabalho Prático 1 para validação de entrada.

 \item \textbf{main.c}\\
	Contém a lógica de execução do programa. Inicia a base de dados e a lista de chaves primárias e depois recebe as opções de operações desejadas pelo usuário, tratando-as e chamando as funções necesśarias em outros arquivos.

 \item \textbf{mem.c, mem.h}\\
	Contém macros e funções para facilitar a alocação de memória para o programa: basicamente, as macros alocam o número desejado de bytes na memória e já abortam o programa em caso de erro. Assim, em vez de
	\texttt{
	\begin{tabbing}
		char* c \= = (char*)malloc(5, sizeof(char)); \\
		if (!c) exit(EXIT\_FAILURE);
	\end{tabbing}
	}

	Tem-se
	\texttt{
	\begin{tabbing}
		char* c = MEM\_ALLOC\_N(char, 5);
	\end{tabbing}
	}

 \item \textbf{menu.c, menu.h}\\
	Continuam a possuir as duas funções existentes no Trabalho Prático 1: exibir uma mensagem inicial e o menu de opções para o usuário, agora com mais opções disponíveis.

 \item \textbf{pk.c, pk.h}\\
	Principal adição deste Trabalho Prático. Responsáveis pela manipulação da estrutura de indexação com chaves primárias. Maiores detalhes seguem depois no relatório.
\end{itemize}

\section{Chaves primárias}\

Decidiu-se utilizar uma estrutura \textit{PrimaryKeyList} para armazenar as chaves primárias composta pelo número atual de registros armazenados (\textit{regnum}), o número máximo de registros que podem ser armazenados (\textit{maxregs}) e uma estrutura interna (\textit{PrimaryKeyRecord}) para armazenamento do número relativo de registro (\textit{rrn}) e o nome da entrada (\textit{name}).

Optou-se por essa estrutura devido à sua facilidade de manutenção e expansão para os futuros trabalhos, caso sejam adicionados novos campos. Inicialmente, utilizou-se apenas um ponteiro para ponteiro que separava número relativo de registro e nome em suas estruturas. Entretanto, a dificuldade de manutenção e utilização dessa estrutura e sua falta de escalabilidade também nos motivaram a buscar algo mais robusto.

A estrutura \textit{PrimaryKeyRecord} é de uso interno das bibliotecas do programa, e o usuário tem contato somente com os campos que ela contém (\textit{rrn} e \textit{name}) quando executa operações em \textit{PrimaryKeyList}.

No header \textbf{pk.h} há definições dos tamanhos dos campos e da própria estrutura de chave primária. Esta contém um campo que é o número relativo do registro(RRN) e outro que é o nome completo da obra(chave primária). Além disso, foi criada uma estrutura que carrega o número de registros carregados na memória, o máximo de registros que podem ser carregados com a alocação de memória corrente, e o vetor de estruturas de chaves primárias. 

Uma decisão importante durante o desenvolvimento deste TP foi o de criar uma estrutura para as chaves primárias. Outra solução encontrada para problemas que tínhamos foi o de criar uma estrutura para carregas o vetor de estruturas de chaves primárias. Assim, fica muito fácil o controle de quantos registros existem e quantos podem existir sem precisar expandir o vetor.

Há funções para diversas finalidads relacionadas às chaves primárias. Inicialização das estruturas, liberação de memória, inserção são algumas. Além dessas, busca por uma chave, carregar o arquivo salvo das chaves primárias, montar as estruturas a partir de uma base de dados e escrever as informações da memória para o disco são outras.

Foi decidido por começar com um vetor de 20 posições para as chaves. Cada vez que esse vetor é preenchido, dobra-se o tamanho do mesmo.

\section{Resultado final}\

Conseguimos fazer um programa sem bugs conhecidos e sem vazamentos de memória encontrados. As consultas e listagem de obras funcionam perfeitamente.

As maiores dificuldades foram encontradas durante a implementaçãodas funções de manipulação de chaves primárias. Desde a escolha da melhor estrutura de dados até a perfeição do sistema de controle de alocação de memória encontramos várias possibilidades diferentes de implementação, e isso caracterizou boa parte do desafio desse Trabalho Prático.

A parte de geração de um arquivo \textit{html} foi de extrema facilidade, sendo que ainda assim causa uma melhoria muito grande na forma como o programa trabalha. O fato de poder ver as imagens num browser ao invés de só ver as informações num terminal deixou o resultado final muito mais interessante.

\end{document}
