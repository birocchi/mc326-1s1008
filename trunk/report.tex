\documentclass{article}
\usepackage[brazil]{babel}
\usepackage[utf8]{inputenc}

\title{\textbf{Relatório TP1}}
\author{\textbf{Grupo 1} \\
  Anderson Phillip Birocchi (072787) \\
  Miguel Francisco Alves de Mattos Gaiowski (076116) \\
  Raphael Kubo da Costa (072201)}
\date{\today}
\begin{document}

\maketitle

\section*{Objetivo}\

Este trabalho prático tem como objetivo implementar um programa que seja capaz de coletar e armazenar dados sobre obras de arte e guardá-los em um arquivo ``.dat'' no disco rígido. Este deve ter sua estrutura dos dados padronizada, resultando num tamanho de arquivo múltiplo de 420 bytes.

Em específico, este trabalho apenas insere registros na base de dados de obras de arte.

\section*{Descrição do Trabalho}\

Definiu-se, inicialmente, dividir o sistema em vários arquivos de modo a facilitar sua manutenção e a escalabilidade. Apesar de o projeto apenas inserir dados na base, posteriormente seu uso crescerá e a divisão em arquivos em vários ``módulos'' facilita o planejamento e a engenharia do software.

O arquivo \textbf{data.h} é onde estão definidos as estruturas de dados usadas no programa. No futuro, quando o projeto crescer, mais tipos serão definidos ali.

Os arquivos \textbf{io.c} e \textbf{io.h} controlam a leitura e escrita de dados entrados pelo usuário. A função readData ``conversa'' com o usuário e chama a função readValues que lê os dados de forma apropriada. A funçao writeData toma conta da escrita dos dados no arquivo.

Convém, aqui, detalhar as funções \textit{stripNewLine} e \textit{readValue}. A validação das entradas do usuário é conhecidamente um problema da linguagem C, como se vê em \textit{http://www.c-faq.com/stdio/index.html}. A função \textit{readValue} recebe dois parâmetros: um \textit{char} onde deve ser armazenada a entrada do usuário e um \textit{size\_t (unsigned int)} com o tamanho máximo da entrada.

Optamos por usar a função fgets para a leitura de dados da entrada padrão, já que podemos especificar o tamanho da entrada. No caso de uso de scanf o usuário poderia, por exemplo, colocar um título com mais de 200 caracteres. Devido ao uso da função fgets, tivemos que criar uma função simplória para extração do `$\backslash$n' no final da string retornada.

Os arquivos menu.c e menu.h carregam as funções de impressão de boas vindas e do menu de opções para o usuário. Como a impressão do menu é algo que ocorre com frequência, é saudável definir funções para tal.

O código contido em main.c contém a lógica de execução do programa, que deve imprimir o menu, sair caso esta seja a opção do usuário, ou inserir uma obra de arte perguntando, logo após, se deseja inserir outro registro.

\section*{Resultado final}\

   Conseguimos fazer um programa funcional, com uma boa interface para ajudar o usuário, e que gera o arquivo contendo os registros no padrão, porém, feito para um usuário que conhece os dados que devem ser entrados, pois a validação dos dados não é perfeita caso um usuário que totalmente desconhece o programa coloque, por exemplo, no campo valor, o ``R\$'', ou no codigo da imagem, o ponto que antecede o tipo da imagem (1200.png). Fora isso o programa não demonstra bugs.


\end{document}
